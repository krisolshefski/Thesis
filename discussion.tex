\chapter{Discussion}  
At the current state, the fine grid method can successfully predict curvature for a select set of geometries. No successful execution of three dimensional flows have been performed or analyzed. Rigorous investigation of the methods outlined above have left the researchers with several fundamental concerns and drawbacks of this method.
\subsection{Summary}
\begin{enumerate}
	\item It is clear that this method, in all iterations, requires well defined heights for successful execution. However, achieving well defined heights may require mesh sizes which are unrealistic for direct numerical simulation. While there may possibly be ways to work around this issue, research thus far has focused on performing successful simulations with geometries that provide ideal mesh size ratios. Investigation into more complex schemes should only be attempted after this has been achieved.
	\item Incorporation of fine grid velocities seemed to have a positive impact on success of the system but required several modifications from the originally proposed method. The addition of a second pressure equation makes this method computationally expensive. So much so that simulation run time can increase by $40\%$. This increased expense contradicts original motivation of the investigation. Modification of the height function method was chosen as implementing the method is straightforward and the cost is low compared to more complex methods. To that point, while the fine grid height function is straightforward to implement, fine grid velocity incorporation and the subsequent incorporation of semi-Lagrangian fluxes are not. 
	\item While little investigation has been done with three dimensional flow fields, accurate and realistic curvature becomes significantly harder to achieve as fitting is now done using planes instead of lines and well defined heights have the ability to drop off in multiple dimensions.
	\item At present, efforts have been considering a simulation which doesn't ``blow up'' to be a success. However, no efforts have gone into investigating the accuracy of this model. This is vital to understanding the merit of the method. 
\end{enumerate} 

\subsection{Future Work}
Continuation of the presented work has the potential to yield a method which could benefit the multiphase community and could be implemented into any in-house solver which utilizes a Rudman dual mesh. Moving forward, it would be best practice to take time to set up a series of checkpoints which could provide a pass/ fail analysis of the current iteration of the fine grid method. A test suite of several different test cases which increase in complexity with each successful completion of a previous case could be an excellent way to gauge whether a method is more or less successful than the previous attempt. Throughout the presented work, we focused on the oscillating droplet test case as our benchmark and if the method ``blew up'', we simply moved on to the next modification. During the final months of this work, we chose to take steps back to more ideal test cases and learned a great deal about the shortcomings of the method as discussed previously. Moving forward, we would recommend the implementation of a unit test suite to apply a more methodical approach to troubleshooting.

There have been many implementations presented in this work. It is possible that some combination of methods which have been described could result in an overall successful scheme. With the previously described test suite, it would be possible to assign metrics to the amount of success of a given combination of methods as the program moves through various complexities of test cases. From here, all combinations could be analyzed and the best combination used to go forward. A truly successful implementation would need to provide realistic curvature calculations for a wide range of flow types and geometries in both 2D and 3D. Developing test cases which range in flow type and dimension could expose further shortcomings of the method with which to investigate.  

\subsection{Conclusion}
The focus of this work was to develop a curvature estimation method which uses information from a mesh that is twice as fine as the prescribed computational grid. This is motivated standard height function methodology which fails to capture perturbations that exist on this fine grid. Left unmitigated, these perturbations can grow unbounded and result in nonphysical dynamics as well as simulation failure. The described method applies a height function onto the fine grid. Additionally, fine grid velocities are introduced to actively reduce the influence of the fine grid perturbations. The proposed method maintains consistent mass and momentum transport and provides accurate interface transport. The method in its current state can only successfully predict curvatures for select geometries. Future work may provide insight into the shortcomings of the method which could be resolved.  However, this work also exposes certain geometric scenarios in which the height function methodology is insufficient. While it can be useful to explore the full breadth of available methods and the possibilities that each provide, it is also important to recognize the fundamental shortcomings that may come with a method. In this work, we believe a fundamental shortcoming of the height function method has been uncovered. 

