% Abstract 
%The abstract must be single spaced and no more than 350 words. The abstract must contain the following elements: (1) statement of the problem, (2) procedure or methods, (3) results, and (4) conclusions. Mathematical formulas, abbreviations, diagrams, and other illustrative materials should not be included. It should be written to be understood by a person who does not have expertise in the field.

Gas-liquid flows can be significantly influenced by the surface tension force, which controls the shape of the interface. The surface tension force is directly proportional to the interface curvature and an accurate calculation of curvature is essential for predictive simulations of these flow types. Furthermore, methods that consistently and conservatively transport momentum, which is discontinuous at the gas-liquid interface, are necessary for robust and accurate simulations. Using a Rudman dual mesh, which discretizes density on a twice as fine mesh, provides consistent and conservative discretization of mass and momentum. The height function method is a common technique to compute an accurate curvature as it is straightforward to implement and provides a second-order calculation. 

When a dual grid is used, the standard height function method fails to capture fine grid interface perturbations and these perturbations can grow. When these growing perturbations are left uncorrected, they can result in nonphysical dynamics and eventual simulation failure. This work extends the standard height function method to include information from the Rudman dual mesh. The proposed method leverages a fine-grid height function method to compute the fine-gird interface perturbations and uses a fine-grid velocity field to oppose the fine-grid perturbations. This approach maintains consistent mass and momentum transport while also providing accurate interface transport that avoids non-physical dynamics. The method is tested using an oscillating droplet test case and compared to a standard height function. Various iterations of the fine grid method are presented and strengths and shortcomings of each are discussed. 