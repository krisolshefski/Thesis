% Abstract 
%The abstract must be single spaced and no more than 350 words. The abstract must contain the following elements: (1) statement of the problem, (2) procedure or methods, (3) results, and (4) conclusions. Mathematical formulas, abbreviations, diagrams, and other illustrative materials should not be included. It should be written to be understood by a person who does not have expertise in the field.

The atomization process is significantly affected by the surface tension force, which controls the size and distribution of droplets. The surface tension force is directly proportional to the interface curvature and an accurate calculation of curvature is essential for predictive simulations of atomization. Furthermore, methods that consistently and conservatively transport momentum, which is discontinuous at the gas-liquid interface, are necessary for robust and accurate simulations. The height function method is a common technique to compute an accurate curvature as it is straightforward to implement and provides a second-order calculation. Additionally, using a Rudman dual mesh (Int. J. Numer. Meth. Fluids, 1998),  which discretizes density on a twice as fine mesh, provides consistent and conservative discretizations of mass and momentum.\\
\\
\indent This work extends the standard height function method to include information from the Rudman dual mesh.  When a dual grid is used, the standard height function method fails to capture fine grid interface perturbations and these perturbations can grow.  The proposed method leverages a fine-grid height function method to compute the fine-gird interface perturbations and uses a fine-grid velocity field to oppose the fine-grid perturbations. This approach maintains consistent mass and momentum transport while also providing accurate interface transport that avoids non-physical dynamics. The method is tested using an oscillating droplet test case and compared to a standard height function. 